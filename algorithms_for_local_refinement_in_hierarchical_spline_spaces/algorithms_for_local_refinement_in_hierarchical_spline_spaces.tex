\documentclass[12pt,a4paper,pdftex]{article}

%\usepackage[notref,notcite]{showkeys}
%\usepackage{latexsym,amsfonts,amssymb,amsthm,amsmath,amscd,eufrak,makeidx}
\usepackage{fullpage}
\usepackage[dvips]{graphicx}
\usepackage{psfrag}
\usepackage{amsmath}
\usepackage{fancybox}
\usepackage{amsfonts}
\usepackage{amsthm}
\usepackage{amssymb}
\usepackage{comment}
\usepackage{cancel}
\usepackage{color}
\usepackage[]{algorithm2e}
\usepackage[]{algorithmicx}
\usepackage{bm}
\usepackage{xspace}
\usepackage[english]{babel}
\usepackage{authblk}

\usepackage[normalem]{ulem} % to be able to use \sout

\usepackage{enumerate}
%\usepackage{pdfsync}

\usepackage[%
%   pdftex=true,  % si se usa pdftex
%   pdfpagemode=UseNone,
%   pdfstartpage=3,   % para empezar en pag. 3
   bookmarks=true,
%   bookmarksopen=true,
%   bookmarksnumbered=true,
%   plainpages=false, % cuando existe p\'agina i y p\'agina 1
%   pdfstartview=XYZ,
%   pdfpagelabels=true, % ii en preambulo
   pagebackref=false, % referencias a citas
   colorlinks,
   linkcolor=blue,
%   linktocpage, % link en el n\'umero de pagina en toc
   anchorcolor=blue,
   citecolor=blue,
   filecolor=blue,
   %pagecolor=blue,
   urlcolor=blue
   ]{hyperref}



 \theoremstyle{plain}
    \newtheorem{theorem}{Theorem}[section]
    \newtheorem{corollary}[theorem]{Corollary}
    \newtheorem{lemma}[theorem]{Lemma}
    \newtheorem{proposition}[theorem]{Proposition}
    \newtheorem{assumption}[theorem]{Assumption}
 \theoremstyle{definition}
    \newtheorem{definition}[theorem]{Definition}
    \newtheorem{remark}[theorem]{Remark}
 \theoremstyle{remark}
  %  \newtheorem{notation}[theorem]{Notation}
    \newtheorem{example}[theorem]{Example}
 
% \newcommand{\Proof}{\noindent\emph{Proof.}\xspace}


% Para el \fint %%%%%%%
%\usepackage{txfonts}
%\newcommand{\fint}{\int}
%%%%%%%%%%%%%%%%%%

%%%%%%%%%%%%%% Paquetes adicionales (sacar despues) %%%%%%%%%%%%%%%
% hyperref
%----------------------------------------------

% definicion de \algoritmo
\newcounter{contadoralgoritmo}
\setcounter{contadoralgoritmo}{0}
\newcommand{\algoritmo}[2]{
\medskip
\begin{center}
\doublebox{ % otras opciones: doublebox, ovalbox, Ovalbox
\begin{minipage}{.95\textwidth}
\refstepcounter{contadoralgoritmo}
\noindent\textbf{ALGORITHM \arabic{contadoralgoritmo}: #1 }
\tt

#2
\end{minipage}
}
\end{center}

}

\newcommand{\Rd}{\color{red}}   

% %%%%%%%% DEFINITION OF MACROS
\newcommand{\note}[1]{{\noindent\centerline{\fbox{\parbox{.9\textwidth}{\textbf{
#1}}}}}}
\newcommand{\bluenote}[1]{{\color{blue}\noindent\centerline{\fbox{\parbox{.9\textwidth}{\textbf{
#1}}}}}}

% %\newcommand{\note}[1]{}
 \newcommand{\snote}[1]{\fbox{\textbf{#1}}}
 %\newcommand{\snote}[1]{}

 \DeclareMathOperator*{\infimum}{inf\phantom{p}\!\!\!}
 \newcommand{\infsup}[2]{\infimum_{#1}\sup_{#2}}


\newcommand{\step}[1]{\noindent\raisebox{1.5pt}[10pt][0pt]{\tiny\framebox{$#1$}}
\xspace}
\newcommand{\D}{\displaystyle}


\DeclareMathOperator{\supp}{supp}
\DeclareMathOperator{\dist}{dist}
\DeclareMathOperator{\Span}{span}
\DeclareMathOperator{\diam}{diam}
\DeclareMathOperator{\Div}{div}
\DeclareMathOperator{\id}{id}
\DeclareMathOperator{\loc}{loc}

\newcommand{\bnew}[1]{{\color{blue}{#1}}}
%\newcommand{\bnew}[1]{#1}
\newcommand{\rdel}[1]{{\color{red}{\sout{#1}}}}
%\newcommand{\rdel}[1]{}

\DeclareMathOperator\mesh{\texttt{MESH}}
\DeclareMathOperator\basis{\texttt{SPACE}}
\DeclareMathOperator\removed{\tt R}
\DeclareMathOperator\marked{\tt MARKED}
\DeclareMathOperator\newcells{\tt NE}
\DeclareMathOperator\markedelements{\tt ME}
\DeclareMathOperator\functionstoremove{\tt MF}
\newcommand\qih{\Pi}
\newcommand\qihh{\breve{\Pi}}
\newcommand\NN{\mathbb N}
%\newcommand\UU{\mathcal U}
\newcommand\AAA{\mathcal A}
\newcommand\BB{\mathcal B}

\newcommand\MM{\mathcal M}
\newcommand\Tau{\mathcal T}
\newcommand\PP{\mathbb P}
\newcommand\QQ{\mathcal Q}
\newcommand\HH{\mathcal H}
%\newcommand\TT{\mathbb T}
\newcommand\VV{\mathcal S}
\newcommand\WW{\mathbb W}
\newcommand\EE{\mathcal E}
\newcommand\RRR{\mathcal R}
\newcommand\SSS{\mathbb S}
\newcommand\CC{\mathcal C}
\newcommand\NNN{\mathcal N}
\newcommand\PPP{\mathcal P}
\newcommand{\JJ}{\mathcal J}
\newcommand{\RR}{\mathbb R}
%\newcommand{\nref}{n}
\newcommand{\h}{h}

\newcommand{\NT}{{\mathcal N}_\Tau}

\newcommand{\etal}{et.~al.}

\newcommand{\norm}[1]{{\left\vert\kern-0.25ex\left\vert\kern-0.25ex\left\vert #1  \right\vert\kern-0.25ex\right\vert\kern-0.25ex\right\vert}}

\newcommand{\bx}{{\bf x}}

\begin{document}

\title{Algorithms for local refinement \\ in hierarchical spline spaces}


\author[1,2,3]{Eduardo M. Garau}%\thanks{Corresponding author}}
\author[1]{Rafael V\'azquez}
\affil[1]{\small Istituto di Matematica Applicata e Tecnologie Informatiche 
`E. Magenes' (CNR), Italy}
\affil[2]{\small Instituto de Matem\'atica Aplicada del Litoral (CONICET-UNL), Argentina}
\affil[3]{\small Facultad de Ingenier\'ia Qu\'imica (UNL), Argentina}

\maketitle
% 
% \begin{abstract}
%  
% \end{abstract}
% 
% \begin{quote}\small
% \textbf{Keywords:} % adaptivity in isogemetric analysis, hierarchical B-splines, \snote{...}
% \end{quote}
% 
% \begin{quote}\small
%  \textbf{Mathematics Subject Classification (2000):}
% 35J15,  % Second-order elliptic equations
% 65N12, % Stability and convergence of numerical methods
% 65N15, % Error bounds
% 65N30, % Finite elements, Rayleigh-Ritz and Galerkin methods, finite methods
% 65N50, % Mesh generation and refinement
% 65Y20  % Complexity and performance of numerical algorithms 
% \end{quote}


\section{Setting}

Let $d\ge 1$. We are going to consider tensor-product $d$-variate spline function spaces on $\Omega:=[0,1]^d\subset \RR^d$, where ${\bf p}:=(p_1,p_2,\dots,p_d)$ denotes the vector of polynomial degrees of the splines with respect to each coordinate direction.

\subsection{Underlying sequence of tensor-product spline spaces}

We consider a given sequence $\{\VV_\ell\}_{n\in\NN_0}$ of tensor-product $d$-variate spline spaces such that
\begin{equation}\label{E:tensor-product spaces}
 \VV_0\subset \VV_1\subset \VV_2\subset\VV_3\subset\dots,
\end{equation}
with the corresponding tensor-product B-spline bases denoted by
\begin{equation}\label{E:tensor-product basis}
 \BB_0,  \BB_1,  \BB_2, \BB_3,\dots,
\end{equation}
respectively. %In order to guarantee~\eqref{E:tensor-product spaces}, we assume that if $\xi$ is a knot in $\Xi_{p_i,n_i^\ell}$ with multiplicity $m$, then $\xi$ is also a knot in $\Xi_{p_i,n_i^{\ell+1}}$ with multiplicity at least $m$, for $i=1,\dots,d$ and $\ell\in\NN_0$. 
Furthermore, for $\ell\in\NN_0$, we denote by $\QQ_\ell$ the tensor-product mesh associated to $\BB_\ell$ and we say that $Q\in\QQ_\ell$ is a \emph{cell of level $\ell$}. We now state some well-known properties of the B-spline basis functions~\cite{deBoor,Schumaker}:
\begin{itemize}
\item \emph{Local linear independence.} For any nonempty open set $O\subset\Omega$, the functions in $\BB_\ell$ that do not vanish identically on $O$, are linearly independent on $O$.
\item \emph{Positive partition of unity.} The B-spline basis functions of level $\ell$ form a partition of the unity 
on $\Omega$, i.e.,
 \begin{equation}\label{E:partition of unity in B_0}
  \sum_{\beta\in\BB_\ell} \beta \equiv 1,\qquad\text{on }\Omega.
 \end{equation}
\item \emph{Two-scale relation between consecutive levels.} The B-splines of level $\ell$ can be written as a linear combination of B-splines of level $\ell+1$. More precisely, 
\begin{equation}\label{E:two scale relation}
 \beta_\ell = \sum_{\beta_{\ell+1}\in\CC(\beta_{\ell})} 
c_{\beta_{\ell+1}}(\beta_\ell)\beta_{\ell+1}, 
\qquad \forall\,\beta_\ell\in\BB_\ell,
\end{equation}
where the coefficients $c_{\beta_{\ell+1}}(\beta_\ell)$ are strictly positive, and $\CC(\beta_\ell)\subset\BB_{\ell+1}$ is the set of children of $\beta_\ell$ as defined in~\cite{BG15}. 
\end{itemize}

\begin{remark}
Notice that if we define $c_{\beta_{\ell+1}}(\beta_\ell):=0$ when $\beta_{\ell+1}$ is not a child of $\beta_\ell$, then equation~\eqref{E:two scale relation} can be written as
\begin{equation}\label{E:two scale relation 2}
 \beta_\ell = \sum_{\beta_{\ell+1}\in\BB_{\ell+1}} 
c_{\beta_{\ell+1}}(\beta_\ell)\beta_{\ell+1}, 
\qquad \forall\,\beta_\ell\in\BB_\ell.
\end{equation}

In particular, we remark that
\begin{equation}\label{E:set of children}
\CC(\beta_\ell)=\{\beta_{\ell+1}\in\BB_{\ell+1}\,|\,c_{\beta_{\ell+1}}(\beta_\ell)>0\}\subset\{\beta_{\ell+1}\in\BB_{\ell+1}\,|\,\supp\beta_{\ell+1}\subset\supp\beta_\ell\}.
\end{equation}

\end{remark}

\subsection{Hierarchical B-spline basis and hierarchical spline space}

\begin{definition} 
If $n\in\NN$, 
we say that ${\bf\Omega}_n := \{\Omega_0,\Omega_1,\dots,\Omega_n\}$ is a 
\emph{hierarchy of subdomains of $\Omega$ of depth $n$} if
\begin{enumerate}
 \item[(i)] $\Omega_\ell$ is the union of cells of level 
$\ell-1$, for $\ell = 1,2,\dots,n$.
 \item[(ii)]$\Omega = \Omega_0 \supset \Omega_1 \supset \dots \supset 
\Omega_{n-1}\supset \Omega_n = \emptyset$.
\end{enumerate} 
\end{definition}

We now define the hierarchical B-spline basis ${\HH}= {\HH}({\bf\Omega}_n)$ by 
% in the following recursive way
% \begin{equation}\label{E:definition of Hltilde}
% \begin{cases}\HH_0:=\BB_0,\\
%  {\HH}_{\ell+1}:=\{\beta\in\HH_{\ell}\,|\,\supp\beta\not\subset\Omega_{\ell+1}\}\cup\{\beta\in\BB_{\ell+1}\,|\, \supp\beta\subset\Omega_{\ell+1}\},\qquad\ell=0,\dots,n-2.\\
%  \HH:=\HH_{n-1}.\end{cases}
% \end{equation}
% If $\RRR_\ell:=\HH_\ell\setminus\HH_{\ell+1}$, then
% $$\RRR_\ell=\{\beta\in\HH_{\ell}\,|\,\supp\beta\subset\Omega_{\ell+1}\}=\{\beta\in\BB_{\ell}\,|\,\supp\beta\subset\Omega_{\ell+1}\}.$$
% Notice that in order to get $\HH_{\ell+1}$ from $\HH_{\ell}$ we replace the set $\RRR_\ell$ by
% $$\{\beta\in\BB_{\ell+1}\,|\, \supp\beta\subset\Omega_{\ell+1}\}.$$
% Moreover, it is easy to check that
\begin{equation}\label{E:Hierarchical basis}
{\HH}= \bigcup_{\ell = 0}^{n-1} \{\beta \in \BB_\ell \mid \supp \beta 
\subset \Omega_\ell \wedge  \supp \beta \not\subset \Omega_{\ell+1}\}.
\end{equation} 
We say that $\beta$ is \emph{active} if $\beta\in\HH$. The 
corresponding underlying mesh $\QQ = \QQ({\bf\Omega}_n)$ is given 
by
\begin{equation}\label{E:hierarchical mesh}
\QQ:= \bigcup_{\ell = 0}^{n-1} \{ Q\in\QQ_\ell\,\mid\, Q\subset \Omega_\ell 
\,\wedge\, Q\not\subset \Omega_{\ell+1}\}, 
\end{equation}
and we say that $Q$ is an \emph{active cell} is $Q\in\QQ$, or that $Q$ is an 
\emph{active cell of level $\ell$} if $Q\in\QQ\cap \QQ_\ell$.


Unlike the B-spline bases $\BB_\ell$ for tensor-product spline spaces, the hierarchical B-spline basis $\HH$ does not constitute a partition of the unity. Instead, in view of the linear independence of functions in $\HH$ and taking into account that $\VV_0=\Span\BB_0\subset\Span\HH$, we have that there exists a set $\{a_\beta\}_{\beta\in\HH}\subset \RR$, uniquely determined, such that
 \begin{equation}\label{E:partition of the unity in the hierarchical space}
\sum_{\beta\in {\HH}} a_\beta \beta \equiv 1, \qquad\text{on } \Omega. 
\end{equation}
 It can be proved that $a_\beta\ge 0$, for all $\beta\in\HH$. On the other hand, we remark that these coefficients depend on the hierarchy of subdomains~${\bf\Omega}_n$.



\section{Refinement of hierarchical spline spaces}\label{S:refinement}

In this section, we present a precise technique to refine locally a given hierarchical spline space $\Span\HH$.


\begin{definition}
 Let ${\bf\Omega}_n := \{\Omega_0,\Omega_1,\dots,\Omega_n\}$ and ${\bf\Omega}_{n+1}^*:= \{\Omega_0^*,\Omega_1^*,\dots,\Omega_n^*,\Omega_{n+1}^*\}$ be hierarchies of subdomains of $\Omega$ of depth (at most) $n$ and $n+1$, respectively. We say that ${\bf\Omega}_{n+1}^*$ is an \emph{enlargement} of ${\bf\Omega}_n$ if
 $$\Omega_\ell\subset\Omega_\ell^*,\qquad \ell=1,2,\dots,n.$$ 
 \end{definition}

 Let $\HH$ and $\QQ$ be the hierarchical B-spline basis and the hierarchical mesh associated to the hierarchy of 
subdomains of depth $n$,
${\bf\Omega}_n := \{\Omega_0,\Omega_1,\dots,\Omega_n\}$.
 
Let ${\bf\Omega}_{n+1}^*$ be an enlargement of ${\bf\Omega}_n$. Now, the corresponding hierarchical B-spline basis ${\HH}^*$ and 
\emph{refined} 
mesh $\QQ^*$ are given by 
\begin{equation}
{\HH}^*:= \bigcup_{\ell = 0}^{n} \{\beta \in \BB_\ell \mid \supp \beta 
\subset \Omega_\ell^* \wedge  \supp \beta \not\subset \Omega_{\ell+1}^*\},
\end{equation}
and
$$\QQ^*:= \bigcup_{\ell = 0}^{n} \{ Q\in\QQ_\ell \,\mid\, Q\subset \Omega_\ell^* 
\,\wedge\, Q\not\subset \Omega_{\ell+1}^*\}.$$



Let $\{a_\beta^*\}_{\beta\in{\HH}^*}$ denote the sequence of coefficients (with respect to the hierarchy ${\bf\Omega}_{n+1}^*$) such that $$\sum_{\beta\in{\HH}^*} a_\beta^*\beta \equiv 1,\qquad\text{ on }\Omega.$$ 



In~\cite{GJS14} has been proved that any enlargement of ${\bf\Omega}_n$ gives rise to a new enriched hierarchical B-spline basis $\HH^*$, in the sense that
$$\Span\HH\subset\Span\HH^*.$$
In order to enlarge the given subdomains ${\bf\Omega}_n = \{\Omega_0,\Omega_1,\dots,\Omega_n\}$ we have to select the areas in $\Omega$ where more ability of approximation is required. Such a choice can be done by selecting to \emph{refine} some active basis functions or some active cells. More precisely, we consider the two following ways of enlarging the hierarchy ${\bf\Omega}_n$:
\begin{itemize}
 \item {\bf Marking basis functions:} We consider a subset $\MM$ of active B-spline basis functions, i.e., $\MM\subset\HH$. Let $\MM_\ell:=\MM\cap\BB_\ell$, for $\ell = 0,1,\dots,n-1$. Now, we define the hierarchy of domains ${\bf\Omega}^*_{n+1} := 
\{\Omega_0^*,\Omega_1^*,\dots,\Omega_n^*,\Omega_{n+1}^*\}$ of depth (at most) 
$n+1$, by
\begin{align}\label{E:marking basis functions}
\begin{cases}\Omega_0^* &:=\Omega_0,\\
\Omega_{\ell}^* &:=\Omega_{\ell}\cup \D\bigcup_{\beta\in\MM_{\ell-1}}\supp 
\beta,\qquad \ell = 1,2,\dots,n,\\
\Omega_{n+1}^*&:= \emptyset.
\end{cases}
\end{align}
Let $\HH^*$ be the hierarchical B-spline basis associated to ${\bf\Omega}^*_{n+1}$. Notice that $\MM\subset\HH\setminus\HH^*$, i.e., at least the functions in $\MM$ have been removed (or deactivated) from the hierarchical basis $\HH$. 
\item {\bf Marking active cells:} We consider a subset $\MM$ of active cells, i.e., $\MM\subset\QQ$. Let $\MM_\ell:=\MM\cap\QQ_\ell$, for $\ell = 0,1,\dots,n-1$. Now, we define the hierarchy of domains ${\bf\Omega}^*_{n+1} := 
\{\Omega_0^*,\Omega_1^*,\dots,\Omega_n^*,\Omega_{n+1}^*\}$ of depth (at most) 
$n+1$, by
\begin{align}\label{E:marking active cells}
\begin{cases}\Omega_0^* &:=\Omega_0,\\
\Omega_{\ell}^* &:=\Omega_{\ell}\cup \D\bigcup_{Q\in\MM_{\ell-1}}Q,\qquad \ell = 1,2,\dots,n,\\
\Omega_{n+1}^*&:= \emptyset.
\end{cases}
\end{align}
Let $\HH^*$ be the hierarchical B-spline basis associated to ${\bf\Omega}^*_{n+1}$ and $\QQ^*$ be the corresponding hierarchical mesh. In this case, $\MM\subset\QQ\setminus\QQ^*$, i.e., all cells in $\MM$ have been refined. 
\end{itemize}


% In the case of hierarchy of subdomains satisfying the support condition, we can 
% \emph{easily} compute the active functions $\beta$ in the finer space having nonzero weight, i.e. 
% $\beta\in\HH^*$, from the knowledge of active functions in the coarse basis $\HH$, provided we have a global numbering for all the basis functions 
% in $\BB_\ell$, for $\ell \in\NN_0$.

\section{Algorithms for initialization and refinement of a hierarchical B-spline basis}


Here we describe an algorithm to compute the active B-splines in the finer basis taking advantage of the knowledge of the active B-splines in the current coarse basis.

We consider a global numbering for all the basis functions in $\BB_\ell$, for each $\ell \in\NN_0$, and assume that we have available the following basic routines related with the underlying tensor-product spline spaces:
\begin{enumerate}
 \item[(1)] Basic routines in each tensor-product space $\VV_\ell$:
\begin{itemize}
 \item $I = \verb+get_cells+ (i_\beta,\ell)$, where $i_\beta$ is the global 
index of a function $\beta\in\BB_\ell$, and $I$ contains the global indices 
of the cells in $\QQ_{\ell}$ which are subsets of $\supp\beta$.
\item $I = \verb+get_neighbors+ (i_\beta,\ell)$, where $i_\beta$ is the global 
index of a function $\beta\in\BB_\ell$, and $I$ contains the global indices 
of functions in $\BB_{\ell}$ whose supports have at least one cell of level $\ell$ within $\supp\beta$.
\item $I = \verb+get_basis_functions+ (i_Q,\ell)$, where $i_Q$ is the global index of a cell $Q\in\QQ_\ell$ and $I$ contains the global indices 
of functions in $\BB_{\ell}$ that do not vanish on $Q$.
\end{itemize}
\item[(2)] Basic routines linking two consecutive levels of the tensor-product spaces ($\VV_\ell$ and $\VV_{\ell+1}$):
\begin{itemize}
\item $I = \verb+split_cell+ (i_Q,\ell)$, where $i_Q$ is the global index of a cell $Q\in\QQ_\ell$ and $I$ contains the global indices 
of the cells in $\QQ_{\ell+1}$ which are inside of $Q$.
\item $[I,c] = \verb+split_fun+ (i_\beta,\ell)$, where $i_\beta$ is the global 
index of a function $\beta\in\BB_\ell$, $I$ contains the global indices of all children of $\beta$, and $c$ 
is an array with the corresponding coefficients given by~\eqref{E:two scale relation}. 
\end{itemize}
\end{enumerate}
\begin{remark}\label{R:two-scale relation}
The routines listed above are in fact elementary. The \emph{hardest} thing in the previous routines is the computation of the coefficients for the two-scale relation~\eqref{E:two scale relation} in the function \texttt{split\_fun}. Nevertheless, by virtue of the tensor-product structure of B-splines, this task can be done by computing the corresponding coefficients in the two-scale relation for univariate B-splines and Kronecker products. On the other hand, it is important to remark that the coefficients in the univariate case can be computed using knot insertion formulae. 
It is important to mention that these coefficients will be used in our algorithms below to compute the coefficients of the hierarchical basis functions for the partition of the unity~(cf.~\eqref{E:partition of the unity in the hierarchical space}). We remark that some a posteriori error estimators may require this information~\cite{BuGa15}.

Finally, we remark that if we do not need the explicit knowledge of the coefficients in~\eqref{E:partition of the unity in the hierarchical space}, we can consider a simple version of the function $\texttt{split\_fun}$ given by 
 $$I = \texttt{split\_fun} (i_\beta,\ell),$$ where $i_\beta$ is the global 
index of a function $\beta\in\BB_\ell$ and $I$ contains the global indices of 
the children of $\beta$. In this case, the algorithms described below can also be considerably simplified. 
\end{remark}


The hierarchical mesh $\QQ$ can be defined through the variable $\mesh = \{E^A_\ell,E^D_\ell\}_{\ell=0}^{n-1}$,  where
\begin{itemize}
 \item $E^A_\ell$ is the array containing the global indices of active cells of level $\ell$, i.e., cells in $\QQ\cap\QQ_\ell$.
 \item $E^D_\ell$ is the array containing the global indices of deactivated cells of level $\ell$, i.e., cells $Q\in\QQ_\ell$ such that $Q\subset\Omega_{\ell+1}$. Notice that $E^D_{n-1}=\emptyset$.
 \end{itemize}
 On the other hand, the hierarchical B-spline basis $\HH$ associated to $\QQ$ can be described through the variable $\basis = \{F^A_\ell,F^D_\ell, W_\ell\}_{\ell=0}^{n-1}$, where
\begin{itemize}
 \item $F^A_\ell$ is the array containing the global indices of active B-splines of level $\ell$, i.e., functions in $\HH\cap\BB_\ell$.
  \item  $F^D_\ell$ is an array containing the global indices of B-splines in $\BB_\ell$ whose supports are subsets of $\Omega_{\ell+1}$, i.e.,
$$F^D_\ell:= \{i_\beta\,|\,\beta\in\BB_\ell\quad\wedge\quad \supp\beta\subset\Omega_{\ell+1}\}.$$
Notice that $F^D_{n-1}=\emptyset$. 
 \item $W_\ell$ is an array containing the values of the coefficients $a_\beta$ for the partition of the unity~\eqref{E:partition of the unity in the hierarchical space} corresponding to the active B-splines $\beta$ of level $\ell$, i.e., to the functions in $F^A_\ell$.
\end{itemize}



\subsection{Getting the new active basis functions from the current ones}
 

Let $\marked=\{M_\ell\}_{\ell=0}^{n-1}$, where $M_\ell\subset F^A_\ell$ (or $M_\ell\subset E^A_\ell$) is the set of global indices of \emph{marked} functions (or elements) of level $\ell$, i.e., functions (or elements) in $\MM_\ell$. Now, we present an algorithm for updating the information in $\mesh$ and $\basis$ when enlarging the hierarchy of subdomains ${\bf \Omega}_n$ with the marked functions or elements as explained in the previous section, cf.~\eqref{E:marking basis functions} and~\eqref{E:marking active cells}.


\begin{algorithm}[H]{$\texttt{function}\quad [\mesh,\basis] = \texttt{refine}(\mesh,\basis,\marked)$} 
\caption{$\texttt{refine}$ (REFINE THE HIERARCHICAL MESH AND SPACE)}
\bigskip
\doublebox{
\begin{minipage}{.95\textwidth}
\medskip

\% This function updates $\mesh$ and $\basis$ when enlarging the current subdomains with the marked functions (or elements) given in $\marked$


\KwData{(Data for $\QQ$, $\HH$ and $\MM$): $\begin{cases}\mesh = \{E^A_\ell,E^D_\ell\}_\ell\\ \basis = \{F^A_\ell, F^D_\ell, W_\ell\}_\ell\\ \marked=\{M_\ell\}_\ell \end{cases}\,\, (\ell=0,1,\dots,n-1).$}
 \KwResult{(Data for $\QQ^*$ and $\HH^*$):  $\begin{cases}\mesh = \{E^A_\ell,E^D_\ell\}_\ell\\ \basis = \{F^A_\ell, F^D_\ell, W_\ell\}_\ell \end{cases}\,\, (\ell=0,1,\dots,n).$}

\medskip

\begin{enumerate}
 \item[] \% REFINE MESH\;
  \item[] \Switch{marked functions or elements}{
  \begin{enumerate}
   \item[1.]\lCase{functions,}{$\markedelements = \texttt{compute\_cells\_to\_refine}(\{E^A_\ell\}_{\ell=0}^{n-1}, \marked)$}
\item[2.]\lCase{elements,}{$\markedelements = \marked$}
\end{enumerate}
}
\item[]\%   $\markedelements$ contains the cells that have to be refined\;
  
\item[3.]$[\mesh, \newcells] = \texttt{refine\_hierarchical\_mesh}(\mesh, \markedelements)$\;
%\item[3.]$[\mesh, \newcells] = \texttt{update\_active\_cells}(\mesh, \markedelements)$\;
 \item[]\% $\newcells$ contains the new cells\;
 \item[]\% REFINE SPACE\;
\item[4.] $\basis = \texttt{refine\_hierarchical\_space}(\mesh, \basis, \marked, \newcells)$\;
\end{enumerate}
 \medskip 
\end{minipage}
}
\end{algorithm}


\subsubsection{Routines for the mesh}

% \begin{algorithm}[H]{$\texttt{function}\quad [\mesh,\newcells] = \texttt{refine\_hierarchical\_mesh}(\mesh, \markedelements)$} 
% \caption{$\texttt{refine\_hierarchical\_mesh}$}
% \bigskip
% \doublebox{
% \begin{minipage}{.95\textwidth}
% \medskip
% \KwData{(Data for $\QQ$ and $\markedelements$): $\begin{cases}\mesh = \{E^A_\ell,E^D_\ell\}_\ell\\  \markedelements = \{\markedelements_\ell\}_\ell \end{cases}\,\, (\ell=0,1,\dots,n-1).$}
%  \KwResult{(Data for $\QQ^*$ and $\newcells$):  $\begin{cases}\mesh = \{E^A_\ell,E^D_\ell\}_\ell\\ \newcells = \{\newcells_\ell\}_\ell\end{cases}\,\, (\ell=0,1,\dots,n).$}
% \medskip
% 
% $[\mesh,\newcells] = \texttt{update\_active\_cells}(\mesh,\markedelements)$\;
% 
%  \end{minipage}
% }
% \end{algorithm}


\begin{algorithm}[H]{$\texttt{function}\quad \markedelements = \texttt{compute\_cells\_to\_refine}(\{E^A_\ell\}_{\ell=0}^{n-1}, \marked)$} 
\caption{$\texttt{compute\_cells\_to\_refine}$}
\bigskip
\doublebox{
\begin{minipage}{.95\textwidth}
\medskip
\% This function computes the indices of cells that have to be splitted when marking for refinement the functions in $\marked$\,
\medskip

 \KwData{$\begin{cases} \{E^A_\ell\}_\ell \text{ (indices of active cells)}\\  \marked = \{M_\ell\}_\ell \end{cases}\,\, (\ell=0,1,\dots,n-1).$}
 \KwResult{$\markedelements = \{\markedelements_\ell\}_{\ell=0}^{n-1}$ (Indices of cells that have to be refined)}
\medskip

 \ForEach{$\ell=0,1,\dots,n-1$}{
 \begin{enumerate}
  \item[1.] Use \texttt{get\_cells} to compute the set of indices $\markedelements_\ell$ of the cells of level $\ell$ that are included in the support of functions in $M_\ell$\;
  \item[2.] $\markedelements_\ell \leftarrow \markedelements_\ell\cap E^A_\ell$ \% Remove the nonactive cells from $\markedelements_\ell$\;
 \end{enumerate}
  }
\medskip 
\end{minipage}
}
\end{algorithm}

\begin{algorithm}[H]{$\texttt{function}\quad [\mesh,\newcells] = \texttt{refine\_hierarchical\_mesh}(\mesh,\markedelements)$} 
\caption{\texttt{update\_active\_cells}}
\bigskip
\doublebox{
\begin{minipage}{.95\textwidth}
\medskip

\% This function updates the active and deactive cells in each level when refining the cells in $\markedelements$

\KwData{(Data for $\QQ$ and $\markedelements$): $\begin{cases}\mesh = \{E^A_\ell,E^D_\ell\}_\ell\\  \markedelements = \{\markedelements_\ell\}_\ell \end{cases}\,\, (\ell=0,1,\dots,n-1).$}
 \KwResult{(Data for $\QQ^*$ and $\newcells$):  $\begin{cases}\mesh = \{E^A_\ell,E^D_\ell\}_\ell\\ \newcells = \{\newcells_\ell\}_\ell\end{cases}\,\, (\ell=0,1,\dots,n).$}
\medskip

\begin{enumerate}
\item[1.] \lIf{$\markedelements_{n-1}\neq \emptyset$,} {$E^A_{n} = E^D_n = \emptyset$}

\item[]\ForEach{$\ell=0,1,\dots,n-1$}{
\begin{enumerate}
 \item[2.]$E^A_\ell \leftarrow E^A_\ell \setminus \markedelements_\ell$ \% Update $E^A_\ell$ by removing the cells to be refined\;
 \item[3.]$E^D_\ell \leftarrow E^D_\ell \cup \markedelements_\ell$ \% Update $E^D_\ell$ by adding the cells that were deactivated\;
 \item[4.] Use \texttt{split\_cell} to get the set of indices $\newcells_{\ell+1}$ of cells of level $\ell+1$ which are inside of the cells in $\markedelements_\ell$\;
 \item[5.]$E^A_{\ell+1} \leftarrow E^A_{\ell+1}\cup \newcells_{\ell+1}$ \% Update $E^A_{\ell+1}$ by adding the new active cells of level $\ell+1$; 
\end{enumerate}
}
\end{enumerate}
\end{minipage}
}
\end{algorithm}

\subsubsection{Routines for the space}

For the routines in this section notice that the variable $\mesh$ is already updated, i.e., it contains the information about $\QQ^*$.


\begin{algorithm}[H]{$\texttt{function}\quad \basis = \texttt{refine\_hierarchical\_space}(\mesh, \basis, \marked, \newcells)$} 
\caption{$\texttt{refine\_hierarchical\_space}$}
\bigskip
\doublebox{
\begin{minipage}{.95\textwidth}
{\footnotesize
\medskip

\KwData{(Data for $\QQ^*$, $\HH$, $\MM$, and $\newcells$): $\begin{cases}\mesh = \{E^A_\ell,E^D_\ell\}_{\ell=0}^{n}\\ \basis = \{F^A_\ell, F^D_\ell, W_\ell\}_\ell\\ \marked=\{M_\ell\}_\ell\\  \newcells = \{\newcells_\ell\}_\ell \end{cases}(\ell=0,\dots,n-1).$}
 \KwResult{(Data for $\HH^*$):  $\basis = \{F^A_\ell, F^D_\ell, W_\ell\}_\ell\quad (\ell=0,1,\dots,n).$}
\medskip
 
 \begin{enumerate}
  \item [1.]$\functionstoremove = \texttt{compute\_functions\_to\_deactivate}(\mesh, \basis, \marked)$\;
  \item[]\% $\functionstoremove$ contains all B-splines that have to be deactivated\;
\item[2.]$\basis=\texttt{update\_active\_functions}(\mesh,\basis,\functionstoremove,\newcells)$\;
 \end{enumerate}
 }
 \end{minipage}
}
\end{algorithm}


\begin{algorithm}[H]{$\texttt{function}\quad \functionstoremove = \texttt{compute\_functions\_to\_deactivate}(\{E^A_\ell\}_{\ell=0}^{n}, \{F^A_\ell\}_{\ell=0}^{n-1},\marked)$} 
\caption{\texttt{compute\_functions\_to\_deactivate}}
\bigskip
\doublebox{
\begin{minipage}{.95\textwidth}
{\footnotesize
\medskip
\% This function computes the indices of the functions that have to be deactivated when marking for refinement the functions (or elements) in $\marked$\,
\medskip

 \KwData{$\begin{cases} \{E^A_\ell\}_{\ell=0}^n \text{ (indices of active cells)}\\ \{F^A_\ell\}_\ell\text{ (indices of active basis functions)}\\ \marked = \{M_\ell\}_\ell \end{cases}\,\, (\ell=0,1,\dots,n-1).$}
 \KwResult{$\functionstoremove = \{\functionstoremove_\ell\}_{\ell=0}^{n-1}$ (Indices of basis functions that have to be deactivated)}
\medskip

 \ForEach{$\ell=0,1,\dots,n-1$}{
 \begin{enumerate}
  \item[] \% Computation of functions which \emph{possibly} have to be removed\;
 \item[]\Switch{marked functions or elements}{
\Case{functions,}{
\begin{enumerate}
 \item[1.] Use \texttt{get\_neighbors} to compute the set of indices $\functionstoremove_\ell$ of functions of level $\ell$ whose supports intersect the support of a function in $M_\ell$\;
 \item[2.] $\functionstoremove_\ell \leftarrow (\functionstoremove_\ell\cap F^A_\ell)\setminus M_\ell$ \% Remove from $\functionstoremove_\ell$ the nonactive functions and the active functions already selected for deactivation\; 
\end{enumerate}
}
 \Case{elements,}{
 \begin{enumerate}
  \item[3.] Use \texttt{get\_basis\_functions} to compute the set of indices $\functionstoremove_\ell$ of functions of level $\ell$ whose supports intersect at least one cell in $M_\ell$\;
  \item[4.] $\functionstoremove_\ell \leftarrow (\functionstoremove_\ell\cap F^A_\ell)$ \% Remove from $\functionstoremove_\ell$ the nonactive functions\;
 \end{enumerate}
}
}
\item[] \% Computation of functions which \emph{in fact} have to be removed\;
\item[5.] Update $\functionstoremove_\ell$ by removing the functions that have al least one active cell of level $\ell$ within their supports. Use \texttt{get\_cells}\;
 \item[6.] \lIf{marking functions,}{$\functionstoremove_\ell =  \functionstoremove_\ell\cup M_\ell $}
 \end{enumerate}
 }
\medskip
}
\end{minipage}
}
\end{algorithm}

\bigskip

\begin{algorithm}[H]{$\texttt{function}\quad \basis = \texttt{update\_active\_functions}(\mesh,\basis,\functionstoremove,\newcells)$\;} 
\caption{\texttt{update\_active\_functions}}
\bigskip
\doublebox{
\begin{minipage}{.95\textwidth}
%\begin{center}
\medskip
{\footnotesize
\% This function updates the active and deactive B-spline basis functions. The inputs $\mesh$ and $\newcells$ are already updated (see \texttt{refine\_hierarchical\_mesh}) and $\functionstoremove$ contains the indices of \emph{all B-splines that have to be deactivated} (see \texttt{compute\_functions\_to\_deactivate}). Finally, the variable $\basis$ in the input contains the information before updating the space\,

\KwData{(Data for $\QQ^*$, $\HH$, $\functionstoremove$ and $\newcells$): $\begin{cases}\mesh = \{E^A_\ell,E^D_\ell\}_{\ell=0}^{n}\\ \basis = \{F^A_\ell, F^D_\ell, W_\ell\}_{\ell=0}^{n-1}\\ \functionstoremove=\{\functionstoremove_\ell\}_{\ell=0}^{n-1}\\  \newcells = \{\newcells_\ell\}_{\ell=0}^n \end{cases}$}
 \KwResult{(Data for $\HH^*$):  $\basis = \{F^A_\ell, F^D_\ell, W_\ell\}_{\ell=0}^n$}
\medskip

\begin{enumerate}
 \item[1.] \lIf{ $\functionstoremove_{n-1}\neq \emptyset$,} {$F^A_n=F^D_n=W_n=\emptyset$}
\item[]\% Update of $\{F^A_\ell,F^D_\ell,W_\ell\}_\ell$\;
\item[]\ForEach{$\ell=0,1,\dots,n-1$}{
\begin{enumerate}
 \item[2.] $\functionstoremove_\ell \leftarrow \functionstoremove_\ell \cup (F^A_\ell\cap F^D_\ell)$ \% This line is important for $\ell\ge 1$\;
\item[3.]$F^A_\ell \leftarrow F^A_\ell \setminus \functionstoremove_\ell$ \% Remove $\functionstoremove_\ell$ from the active functions of level $\ell$\;
\item[4.]Save in \texttt{W} the values of $W_\ell$ corresponding to functions in $\functionstoremove_\ell$ and update $W_\ell$ by removing these values\;
 \item[5.]$F^D_\ell\leftarrow F^D_\ell\cup \functionstoremove_\ell$ \% Update $F^D_\ell$ by adding the functions to be deactivated\;
\item[]\ForEach{$\beta$ in $\functionstoremove_\ell$}{
\begin{enumerate}
  \item [6.] Compute the set of indices $I_\beta$ of the B-splines of level $\ell+1$ and the corresponding coefficients $c$ of the two-scale relation~\eqref{E:two scale relation} when writing $\beta$ as a linear combination of functions in $\BB_{\ell+1}$ 
 \item[7.] Use \texttt{get\_cells} to update $F^D_{\ell+1}$ by adding the functions in $(I_\beta\setminus F^A_{\ell+1})\setminus F^D_{\ell+1}$ that have no active cell of level $\ell+1$ within its support.
 \item[8.] $F^A_{\ell+1}\leftarrow F^A_{\ell+1}\cup I_\beta$ \% Enlarge $F^A_{\ell+1}$ by adding the \emph{possible} new active functions\;
 \item[9.] Enlarge $W_{\ell+1}$ in order to match with the new $F^A_{\ell+1}$, setting equal to zero the coefficients corresponding to the new functions in $F^A_{\ell+1}$.
  \item[] \% Now, we update the values in $W_{\ell+1}$
  \item[10.] \lForEach{ $\beta_{\ell+1}$ in $I_\beta$}{
  $a_{\beta_{\ell+1}}\leftarrow a_{\beta_{\ell+1}}+a_\beta*c_{\beta_{\ell+1}}$, where $a_\beta$ is the value in \texttt{W} corresponding to $\beta$, and $a_{\beta_{\ell+1}}$ and $c_{\beta_{\ell+1}}$ are the values in $W_{\ell+1}$ and $c$, respectively, corresponding to $\beta_{\ell+1}$}
 \end{enumerate}
}
\item[]\% Now, we activate B-splines of level $\ell+1$ that are not children of any deactivated B-spline of level $\ell$\;
\item[11.] Use \texttt{get\_basis\_functions} to compute the indices $I$ of the B-splines of level $\ell+1$ that do not vanish in some cell of $\newcells_{\ell+1}$\;
\item[12.] $I\leftarrow I\setminus F^A_{\ell+1}$ \% Remove from $I$ the functions which were already active\;
\item[13.] Use \texttt{get\_cells} to update $I$ by removing the functions such that have at least one cell of level $\ell+1$ in their support that does not belong to $E^A_{\ell+1}\cup E^D_{\ell+1}$\;
\item[14.] $F^A_{\ell+1}\leftarrow F^A_{\ell+1}\cup I$\;
\item [15.] Enlarge $W_{\ell+1}$ in order to match with the new $F^A_{\ell+1}$, setting equal to zero the coefficients corresponding to the new functions in $F^A_{\ell+1}$\;
\end{enumerate}
}
\end{enumerate}
\medskip
}%\end{center}
\end{minipage}
}
\end{algorithm}


%%%%%%%%%%%%%%%%%%%%%%%%%%%%%%%%%%%%%%%%%%%%%%%%%%%%%%%%%%%
\newpage

\begin{algorithm}[H]{$\texttt{function}\quad \basis = \texttt{update\_active\_functions}(\mesh,\basis,\functionstoremove,\newcells)$\;} 
\caption{\texttt{update\_active\_functions}}
\bigskip
\doublebox{
\begin{minipage}{.95\textwidth}
%\begin{center}
\medskip
{\footnotesize
\% This function updates the active and deactivated B-spline basis functions, {\Rd when using the simplified hierarchical space}. The inputs $\mesh$ and $\newcells$ are already updated (see \texttt{refine\_hierarchical\_mesh}) and $\functionstoremove$ contains the indices of \emph{all B-splines that have to be deactivated} (see \texttt{compute\_functions\_to\_deactivate}). Finally, the variable $\basis$ in the input contains the information before updating the space\,

\KwData{(Data for $\QQ^*$, $\HH$, $\functionstoremove$ and $\newcells$): $\begin{cases}\mesh = \{E^A_\ell,E^D_\ell\}_{\ell=0}^{n}\\ \basis = \{F^A_\ell, F^D_\ell\}_{\ell=0}^{n-1}\\ \functionstoremove=\{\functionstoremove_\ell\}_{\ell=0}^{n-1}\\  \newcells = \{\newcells_\ell\}_{\ell=0}^n \end{cases}$}
 \KwResult{(Data for $\HH^*$):  $\basis = \{F^A_\ell, F^D_\ell\}_{\ell=0}^n$}
\medskip

\begin{enumerate}
 \item[1.] \lIf{ $\functionstoremove_{n-1}\neq \emptyset$,} {$F^A_n=F^D_n=\emptyset$}
\item[]\ForEach{$\ell=0,1,\dots,n-1$}{
\begin{enumerate}
\item[2.]$F^A_\ell \leftarrow F^A_\ell \setminus \functionstoremove_\ell$ \% Remove $\functionstoremove_\ell$ from the active functions of level $\ell$\;
\item[3.]$F^D_\ell\leftarrow F^D_\ell\cup \functionstoremove_\ell$ \% Update $F^D_\ell$ by adding the functions to be deactivated\;
\item[4.] $F^C \leftarrow$ \texttt{split\_fun} $(\functionstoremove_\ell)$, \% Compute the list of children functions of the marked functions;
\item[5.] $F^C \leftarrow F^C \setminus (F^A_{\ell+1} \cup F^D_{\ell+1})$ Take only the children that have not been already activated.
 \item[6.] $F^A_{\ell+1}\leftarrow F^A_{\ell+1}\cup F^C$ \% Enlarge $F^A_{\ell+1}$ by adding the new active functions\;
 \item[7.] Use \texttt{get\_cells} to update $\functionstoremove_{\ell+1}$ by adding the functions in $F^C$ that have no active cell of level $\ell+1$ within its support.

\end{enumerate}
}
\end{enumerate}
\medskip
}%\end{center}
\end{minipage}
}
\end{algorithm}


%%%%%%%%%%%%%%%%%%%%%%%%%%%%%%%%%%%%%%%%%%%%%%%%%%%%%%%%%%%
\newpage

\begin{algorithm}[H]{$\texttt{function}\quad \basis = \texttt{update\_active\_functions}(\mesh,\basis,\functionstoremove,\newcells)$\;} 
\caption{\texttt{update\_active\_functions}}
\bigskip
\doublebox{
\begin{minipage}{.95\textwidth}
%\begin{center}
\medskip
{\footnotesize
\% This function updates the active and deactivated B-spline basis functions, {\Rd when using the standard hierarchical space}. The inputs $\mesh$ and $\newcells$ are already updated (see \texttt{refine\_hierarchical\_mesh}) and $\functionstoremove$ contains the indices of \emph{all B-splines that have to be deactivated} (see \texttt{compute\_functions\_to\_deactivate}). Finally, the variable $\basis$ in the input contains the information before updating the space\,

\KwData{(Data for $\QQ^*$, $\HH$, $\functionstoremove$ and $\newcells$): $\begin{cases}\mesh = \{E^A_\ell,E^D_\ell\}_{\ell=0}^{n}\\ \basis = \{F^A_\ell, F^D_\ell\}_{\ell=0}^{n-1}\\ \functionstoremove=\{\functionstoremove_\ell\}_{\ell=0}^{n-1}\\  \newcells = \{\newcells_\ell\}_{\ell=0}^n \end{cases}$}
 \KwResult{(Data for $\HH^*$):  $\basis = \{F^A_\ell, F^D_\ell\}_{\ell=0}^n$}
\medskip

\begin{enumerate}
 \item[1.] \lIf{ $\newcells_{n-1}\neq \emptyset$,} {$F^A_n=F^D_n=\emptyset$}
\item[]\ForEach{$\ell=0,1,\dots,n-1$}{
\begin{enumerate}
\item[2.]$F^A_\ell \leftarrow F^A_\ell \setminus \functionstoremove_\ell$ \% Remove $\functionstoremove_\ell$ from the active functions of level $\ell$\;
\item[3.]$F^D_\ell\leftarrow F^D_\ell\cup \functionstoremove_\ell$ \% Update $F^D_\ell$ by adding the functions to be deactivated\;
\item[4.] $F^C \leftarrow$\texttt{get\_basis\_functions($\newcells_{\ell+1}$)} \% Compute the indices of B-splines of level $\ell+1$ that do not vanish in some cell of $\newcells_{\ell+1}$\;
\item[5.] $F^C \leftarrow F^C \setminus F^A_{\ell+1}$ \% Remove from $F^C$ the functions that were already active\;
\item[6.] Use \texttt{get\_cells} to update $F^C$ by removing the functions such that have at least one cell of level $\ell+1$ in their support that does not belong to $E^A_{\ell+1}\cup E^D_{\ell+1}$\;
\item[7.] $F^A_{\ell+1}\leftarrow F^A_{\ell+1}\cup F^C$\;
\end{enumerate}
}
\end{enumerate}
\medskip
}%\end{center}
\end{minipage}
}
\end{algorithm}

\newpage

\subsection{Initialization of a hierarchical spline space}

The function \texttt{refine} detailed in the previous section can be also used to select the active functions of a hierarchical B-spline basis $\HH$, if we know the active cells of each level~$\{E^A_\ell\}_{\ell=0}^{n-1}$. 

\begin{algorithm}[H]{$\texttt{function}\quad [\mesh,\basis] = \texttt{build\_hierarchical\_space}(\{E^A_\ell\}_{\ell=0}^{n-1})$} 
\caption{$\texttt{build\_hierarchical\_space}$}
\bigskip
\doublebox{
\begin{minipage}{.95\textwidth}
\medskip

\KwData{$\{E^A_\ell\}_{\ell=0}^{n-1}$ (Active elements in each level)}
 \KwResult{(Data for $\QQ$ and $\HH$):  $\begin{cases}\mesh = \{E^A_\ell,E^D_\ell\}_\ell\\ \basis = \{F^A_\ell, F^D_\ell, W_\ell\}_\ell \end{cases}\,\, (\ell=0,1,\dots,n-1).$}
\medskip

\begin{enumerate}
 \item[1.] Define $\hat{E}^A_0$ as the set of indices of all the cells in $\QQ_0$\;
 \item[2.] Define $F^A_0$ as the set of indices of all the basis functions in $\BB_0$\;

\item[3.] Define the corresponding array of weights $W_0$ setting all the values equal to $1$\;
\item[4.] $\hat{E}^D_0=\emptyset$\;
\item[5.] $F^D_0=\emptyset$\;
\item[6.] $\mesh=\{\hat{E}^A_0,\hat{E}^D_0\}$\; 
\item[7.] $\basis=\{F^A_0,F^D_0,W_0\}$\;
\item[] \ForEach{$k = 0,\dots,n-2$}{
\begin{enumerate}
 \item[8.] Let $M_0 =\dots = M_{k-1}=\emptyset$ and $M_{k} = \hat{E}^A_k\setminus E^A_k$. Then, $\marked=\{M_\ell\}_{\ell=0}^k$\;
 \item[9.] $[\mesh,\basis]\,=\,\texttt{refine}(\mesh,\basis,\marked)$\;
\end{enumerate}
 }
\end{enumerate}









\medskip
\end{minipage}
}
\end{algorithm}



\begin{remark}
 In order to make the last algorithm more understandable we have used the function $\verb+refine+$. This algorithm can be improved by adapting the code of $\verb+refine+$ instead of call it. Notice that the set of marked cells in each of the calls to $\verb+refine+$ has cells only of one level. 
\end{remark}


%\note{ $C_\ell$ is a connectivity matrix which cointains the global indices of active functions whose support intersect $Q$, for every $Q\in\QQ\cap\QQ_\ell$. Here, we also need to store the corresponding levels together with the global indices of active functions.}

\section{Assembly of the matrix}
{\Rd SOME NOTATION, to be changed after discussion:
\begin{itemize}
\item $N_\ell$: number of basis functions in $\BB_\ell$.
\item $A_\ell$: number of active functions of level $\ell$. Give a name to the sets in \eqref{E:Hierarchical basis}.
\item $\tilde A_\ell := \sum_{k=0}^\ell A_\ell$, number of active functions up to level $\ell$.
\item $\beta^A_{j,\ell}$ active functions. Probably we can avoid these.
\end{itemize}
}

One of the issues when implementing IGA with hierarchical splines is the assembly of the matrices. In order to compute the matrices of the discrete problem it is necessary to evaluate integrals that involve basis functions from different levels, and possibly in a level that does not correspond to the level of the functions. For instance, to compute one entry of the mass matrix, one must compute
\begin{equation*}
\int_\Omega \beta_{i,\ell_i} \beta_{j,\ell_j} \, d\bx = \sum_{Q \in \QQ} \int_Q \beta_{i,\ell_i} \beta_{j,\ell_j} \, d\bx = \sum_{\ell = \max\{\ell_i, \ell_j\}}^n \sum_{Q_\ell \in \QQ_\ell} \int_{Q_\ell} \beta_{i,\ell_i} \beta_{j,\ell_j} \, d\bx.
\end{equation*}

In order to compute the integrals, we take advantage of the two-scale relation \eqref{E:two scale relation}: {\Rd I AM CHANGING THE NOTATION}
\begin{equation*}
 \beta_{i,\ell} = \sum_{k=1}^{N_{\ell+1}}
c_{k,\ell+1}(\beta_{i,\ell})\beta_{k,\ell+1}, 
\qquad \forall\,\beta_{i,\ell} \in\BB_\ell,
\end{equation*}
and computing the coefficients for each basis function, we obtain an operator $C_\ell^{\ell+1}: \VV_\ell \longrightarrow \VV_{\ell+1}$, which can be written in matrix form.

Succesively applying the two-scale relation, we obtain the general version
\begin{equation*}
 \beta_{i,\ell} = \sum_{k=1}^{N_{\ell+m}}
c_{k,\ell+m}(\beta_{i,\ell})\beta_{k,\ell+m}, 
\qquad \forall\,\beta_{i,\ell} \in\BB_\ell,
\end{equation*}
and the matrix operator $C_\ell^{\ell+m} = C_{\ell+m-1}^{\ell+m} \ldots C_{\ell+1}^{\ell+2} C_\ell^{\ell+1}$.

Plugging this expression into our integral, and after reordering, we obtain that 
\begin{align*}
\int_\Omega \beta_{i,\ell_i} \beta_{j,\ell_j} \, d\bx = \sum_{\ell = \max\{\ell_i, \ell_j\}}^n \sum_{Q_\ell \in \QQ_\ell}  \int_{Q_\ell} \left(\sum_{k_i=1}^{N_\ell} c_{k_i,\ell}(\beta_{i,\ell_i}) \beta_{k_i,\ell}\right) \left(\sum_{k_j=1}^{N_\ell} c_{k_j,\ell}(\beta_{j,\ell_j}) \beta_{k_j,\ell}\right) \, d\bx = \\
\sum_{\ell = \max\{\ell_i, \ell_j\}}^n \sum_{k_i=1}^{N_\ell} \sum_{k_j=1}^{N_\ell} c_{k_i,\ell}(\beta_{i,\ell_i}) c_{k_j,\ell}(\beta_{j,\ell_j}) \sum_{Q_\ell \in \QQ_\ell}  \int_{Q_\ell}  \beta_{k_i,\ell}  \beta_{k_j,\ell} \, d\bx.
\end{align*}

After the arrangements, we see that it is only needed to compute, in the active elements of level $\ell$, the integrals involving the tensor product functions of that level, both active and inactive. The computation of these integrals is easily available in any software for IGA. 

The only missing part is the computation of the coefficients in an efficient way. This can be done rewriting the equations in matrix form. Let us denote by $M_\ell$ the matrix obtained from function of level $\ell$, that is
\begin{equation*}
(M_\ell)_{k_i k_j} = \sum_{Q_\ell \in \QQ_\ell}  \int_{Q_\ell}  \beta_{k_i,\ell}  \beta_{k_j,\ell} \, d\bx,
\end{equation*}
and let us also denote by $\tilde I_\ell \in {\cal M}_{N_\ell, A_\ell}$ the rectangular matrix such that 
$$
(\tilde I_\ell)_{ij} = \left \{ 
\begin{array}{l}
1, \text{ if }  \beta_{i,\ell} = \beta^A_{j,\ell} (active)\\
0, \text{ otherwise. }
\end{array}
\right.$$

We define the matrices for basis change $C_\ell \in {\cal M}_{N_\ell, \tilde A_\ell}$ by the recursive algorithm
\begin{enumerate}
\item $C^0 = \tilde I_0$.
\item $C_\ell = [C_{\ell-1}^\ell C_{\ell-1}, \tilde I_\ell]$.
\end{enumerate}

By doing so, the global mass matrix is written as
\begin{equation*}
M = \sum_{\ell = 0}^n C^T_\ell M_\ell C_\ell.
\end{equation*}

%===============================================================================
\begin{thebibliography}{1}
%===============================================================================
\bibitem[BG15a]{BG15}
A. Buffa and E.M. Garau, \emph{New refinable spaces and local approximation estimates for hierarchical splines}, submitted, 2015.

\bibitem[BG15b]{BuGa15}
A. Buffa and E.M. Garau, \emph{A posteriori error estimators for hierarchical B-spline discretizations}, in preparation, 2015.

% 
% \bibitem[BG15]{BG15}
% A. Buffa and C.~Giannelli, \emph{Adaptive isogemetric techniques via hierarchical splines: error 
% estimator and convergence}, submitted, 2015.
% 
% \bibitem[CKNS08]{CKNS08}
% J.M. Cascon, C. Kreuzer, R.H. Nochetto and K.G. Siebert, \emph{Quasi-optimal convergence rate for an adaptive finite element method}, SIAM J. Numer. Anal. 46 (2008), no. 5, 2524--2550. 
% 
% \bibitem[CW96]{Chua-Wheeden}
% S.-K. Chua and R. L. Wheeden, \emph{Estimates of best constants for weighted 
% Poincar\'e inequalities on convex domains}, Proceedings of LMS \textbf{93} 
% (2006), 197--226.

\bibitem[dB01]{deBoor}
C. de Boor, \emph{A practical guide to splines}, Revised edition. Applied Mathematical Sciences, 27. Springer-Verlag, New York, 2001. 

\bibitem[GJS14]{GJS14}
C. Giannelli, B. J\"uttler, and H. Speleers, \emph{Strongly stable bases for adaptively refined multilevel spline spaces}, Adv. Comput. Math. 40 (2014), no. 2, 459--490.

% \bibitem[K97]{Kraft}
% R. Kraft, \emph{Hierarchical B-splines}, Preprint of Math. Inst. A Uni.Stuttgart.

\bibitem[K98]{Kraft-thesis}
R. Kraft, \emph{Adaptive und linear unabh\"angige multilevel B-Splines und ihre Anwendungen}, Ph.D. thesis, Universit\"at Stuttgart, 1998. 

% 
% \bibitem[MSV08]{MSV-convergence}
% P.~Morin, K.G.~Siebert and A.~Veeser, \emph{A basic convergence
%   result for conforming adaptive finite elements}, Math. Models Methods Appl.
%   Sci. \textbf{18} (2008), no.~5, 707--737.
% 
% 
% \bibitem[NV]{Nochetto-Veeser}
% R.H. Nochetto and A. Veeser, \emph{Primer of adaptive finite element 
% methods}, Multiscale and Adaptivity: Modeling, Numerics and Applications, CIME 
% Lectures, eds R. Naldi and G. Russo, Springer (to appear).

\bibitem[S07]{Schumaker}
L.L. Schumaker, \emph{Spline functions: basic theory}, Third edition, Cambridge Mathematical Library, Cambridge University Press, Cambridge, 2007.

% 
% \bibitem[S11]{S11}
% K.G.~Siebert, \emph{A convergence proof for adaptive finite elements without lower bound}, IMA J. Numer. Anal. 31 (2011), no. 3, 947--970. 

% \bibitem[SM14]{SM14}
% H. Speleers and C. Manni, \emph{Effortless quasi-interpolation in hierarchical spaces}, Preprint, 2014.
% 
% \bibitem[VV09]{Veeser-Verfurth}
% A. Veeser and R. Verf\"urth, \emph{Explicit upper bounds for dual norms of 
% residuals}, SIAM J. Numer. Anal. 47 (2009), no. 3, 2387--2405.


\bibitem[VGJS11]{Vuong etal}
A.-V. Vuong, C. Giannelli, B. J\"uttler and B. Simeon, \emph{A hierarchical approach to adaptive local refinement in isogeometric analysis}, Comput. Methods Appl. Mech. Engrg. 200 (2011), no. 49-52, 3554--3567.


\end{thebibliography}




\end{document}
