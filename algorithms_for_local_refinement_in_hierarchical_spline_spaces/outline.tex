\documentclass[a4paper]{siamltex1213}

\usepackage[dvips]{graphicx}
\usepackage{amsmath}
\usepackage{fancybox}
\usepackage{amsfonts}
\usepackage{amssymb}
\usepackage{color}
\usepackage{algorithm}
\usepackage{algpseudocode}

\newcommand{\Rd}{\color{red}}   

\title{DECIDE THE TITLE}

\author{Eduardo M.~Garau\footnotemark[1]\ \footnotemark[2]\ \footnotemark[3] \and Rafael V\'azquez\footnotemark[1]}

\begin{document}
\maketitle

\renewcommand{\thefootnote}{\fnsymbol{footnote}}
\footnotetext[1]{Istituto di Matematica Applicata e Tecnologie Informatiche `E. Magenes' (CNR), Italy}
\footnotetext[2]{Instituto de Matem\'atica Aplicada del Litoral (CONICET-UNL), Argentina}
\footnotetext[3]{Facultad de Ingenier\'ia Qu\'imica (UNL), Argentina}
\renewcommand{\thefootnote}{\arabic{footnote}}

\section{Introduction}
We give the algorithms for adaptivity, our focus is on the SOLVE and REFINE modules. Discuss the differences with existing references. The paper comes with an open-source and free implementation in GeoPDEs.

\section{Mathematical definitions}
We should try to avoid defining things for tensor product spline spaces.
\subsection{Hierarchical B-splines}
Definition as in current section 2.1 + 2.2. Notation for basis, basis functions, number of functions per level. We can try to define things in such a way that active and deactivated functions appear in the algorithm.

Two-scale relation. Matrix operator. Partition of unity.
\subsection{Simplified hierarchical B-splines (Can we find a better name?)}
Definition. Two-scale relation. Partition of unity. In general, the results of the paper work for both spaces. We will use the $\tilde H$ when they differ.

\subsection{Truncated hierarchical B-splines}
Definition. Spanned space. Two-scale relation and partition of unity.

\subsection{Refinement of hierarchical splines}
Enlargement as defined in Definition~3.1. Span H $\subset$ Span H$^*$. Refinement matrix between the two-spaces (current Section 4.2).

Marking strategies: we can mark either functions or elements.

Refinement strategies: mention the admissible meshes by Annalisa and Carlotta.


\section{Classes/structures for the implementation of HB-splines}
Introduce them in a general way. What is the information they should contain. What are the needed methods.

\subsection{The basic structures}
The Cartesian grid and the tensor-product space. They should contain the functions to compute the connectivity, but also \texttt{GetCells, GetNeighbors, GetBasisFunctions}. 

We also need something to compute information between two consecutive levels: \texttt{GetChildren} (of elements and of functions). Computation of the matrix $C_\ell^{\ell+1}$. Cite Casciola-Romani.
\subsection{Hierarchical mesh}
It must contain: 
\begin{itemize}
\item the current number of levels $n$.
\item a Cartesian grid for each level. 
\item the list of active elements per level.
\item the list of deactivated elements per level.
\end{itemize}
And it should also contain the procedures to compute the needed information in the set of active elements. This can be easily get from the Cartesian grids.

\subsection{Hierarchical space}
It must contain: 
\begin{itemize}
\item the current number of levels $n$.
\item a tensor-product space for each level. 
\item the list of active functions per level.
\item the list of deactivated functions per level.
\item The coefficients of the two-scale relation between consecutive levels, in 1D.
\end{itemize}

\subsection{Boundary}
The boundary of a hierarchical mesh is a hierarchical mesh. The boundary of a hierarchical space is  a hierarchical space. One can take advantage of this fact to implement things.

\subsection{{\Rd Anything about multipatch?}}

\section{The SOLVE module}
Assembly of matrices and vectors
Current section 5 in the paper.

\section{The REFINE module}
The algorithms in the draft, written in a clear and ordered way.

Should we add something about the refinement strategy by Annalisa and Carlotta?

{\Rd I vote for add something about COARSENING. Maybe just marking elements.}

\section{Implementation in GeoPDEs}
The tensor product spaces already exist. We have implemented this BLABLABLA
\subsection{Hierarchical mesh}
List all the properties and methods inside the class.
\subsection{Hierarchical space}
List all the properties and methods inside the class.
\subsection{Assembly}

\section{Numerical tests}
L-shaped domain. Fichera's corner? Checkerboard pattern? Can we look for a more difficult problem (not the Laplacian)?

Coarsening?

\end{document}
